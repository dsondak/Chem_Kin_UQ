\usepackage[bottom=10em]{geometry} % Reduces the whitespace at the bottom of the page so more text can fit

\usepackage[english]{babel} % English language
\usepackage{lipsum} % Used for inserting dummy 'Lorem ipsum' text into the template

\usepackage[utf8]{inputenc} % Uses the utf8 input encoding
\usepackage[T1]{fontenc} % Use 8-bit encoding that has 256 glyphs

\usepackage[osf]{mathpazo} % Palatino as the main font
\linespread{1.05}\selectfont % Palatino needs some extra spacing, here 5% extra
\usepackage[scaled=.88]{beramono} % Bera-Monospace
\usepackage[scaled=.86]{berasans} % Bera Sans-Serif
\usepackage[euler-digits]{eulervm}

\usepackage{booktabs,array} % Packages for tables

\usepackage{amsmath} % For typesetting math
\usepackage{graphicx} % Required for including images
\usepackage{etoolbox}
\usepackage[norule]{footmisc} % Removes the horizontal rule from footnotes
\usepackage{lastpage} % Counts the number of pages of the document

\usepackage{natbib}

\usepackage[dvipsnames]{xcolor}  % Allows the definition of hex colors
\definecolor{titleblue}{rgb}{0.16,0.24,0.64} % Custom color for the title on the title page
\definecolor{linkcolor}{rgb}{0,0,0.42} % Custom color for links - dark blue at the moment

\DeclareFixedFont{\textcap}{T1}{phv}{bx}{n}{1.5cm} % Font for main title: Helvetica 1.5 cm
\DeclareFixedFont{\textaut}{T1}{phv}{bx}{n}{0.8cm} % Font for author name: Helvetica 0.8 cm

\usepackage[nouppercase,headsepline]{scrpage2} % Provides headers and footers configuration
\pagestyle{scrheadings} % Print the headers and footers on all pages
\clearscrheadfoot % Clean old definitions if they exist

\automark[chapter]{chapter}
\ohead{\headmark} % Prints outer header

\setlength{\headheight}{25pt} % Makes the header take up a bit of extra space for aesthetics
\setheadsepline{.4pt} % Creates a thin rule under the header

\ofoot[\normalfont\normalcolor{\thepage\ |\  \pageref{LastPage}}]{\normalfont\normalcolor{\thepage\ |\  \pageref{LastPage}}} % Creates an outer footer of: "current page | total pages"

% For color boxes
\usepackage{tcolorbox}

% For chemistry
\usepackage{mhchem}

% Hyperlink configuration
\usepackage[
    pdfauthor={}, % Your name for the author field in the PDF
    pdftitle={Laboratory Journal}, % PDF title
    pdfsubject={}, % PDF subject
    bookmarksopen=true,
    linktocpage=true,
    urlcolor=linkcolor, % Color of URLs
    citecolor=linkcolor, % Color of citations
    linkcolor=linkcolor, % Color of links to other pages/figures
    backref=page,
    pdfpagelabels=true,
    plainpages=false,
    colorlinks=true, % Turn off all coloring by changing this to false
    bookmarks=true,
    pdfview=FitB]{hyperref}

\usepackage[stretch=10]{microtype} % Slightly tweak font spacing for aesthetics

%\setlength\parindent{0pt} % Uncomment to remove all indentation from paragraphs

%----------------------------------------------------------------------------------------
%       NEW COMMANDS
%----------------------------------------------------------------------------------------

\newcommand{\lr}[1]{\left(#1\right)}
\newcommand{\pdeone}[2]{\frac{\partial #1}{\partial #2}}
\newcommand{\pden}[3]{\frac{\partial^{#3} #1}{\partial #2^{#3}}}
\newcommand{\odeone}[2]{\frac{\mathrm{d} #1}{\mathrm{d} #2}}
\newcommand{\nup}[1]{\nu_{#1}^{\prime\prime}}
\newcommand{\nur}[1]{\nu_{#1}^{\prime}}
\newcommand{\frate}[1]{K_{#1}^{\left(f\right)}}
\newcommand{\rrate}[1]{K_{#1}^{\left(r\right)}}

\newcommand{\nrd}{M_{D}}
\newcommand{\nsd}{N_{D}}
\newcommand{\xd}{\mathbf{x}^{D}}
\newcommand{\nrr}{M_{R}}
\newcommand{\nsr}{N_{R}}
\newcommand{\na}{N_{a}}
\newcommand{\xr}{\mathbf{x}^{R}}
\newcommand{\x}{\mathbf{x}}

\newcommand{\rrd}{\dot{\boldsymbol{\omega}}^{D}}
\newcommand{\rrr}{\dot{\boldsymbol{\omega}}^{R}}
\newcommand{\rr}{\dot{\boldsymbol{\omega}}}
\newcommand{\hd}{\mathbf{h}^{D}}
\newcommand{\hr}{\mathbf{h}^{R}}
\newcommand{\h}{\mathbf{h}}
\newcommand{\hc}{\mathbf{h}^{\prime}}
\newcommand{\cpd}{\mathbf{c}_{p}^{D}}
\newcommand{\cpr}{\mathbf{c}_{p}^{R}}
\newcommand{\cp}{\mathbf{c}_{p}}
\newcommand{\htc}{h^{\prime}}
\newcommand{\htd}{h^{D}}
\newcommand{\hic}[1]{h_{#1}^{\prime}}
\newcommand{\hid}[1]{h_{#1}^{D}}
\newcommand{\hir}[1]{h_{#1}^{R}}
\newcommand{\xid}[1]{x_{#1}^{D}}
\newcommand{\xir}[1]{x_{#1}^{R}}

\newcommand{\Sop}{\boldsymbol{\mathcal{S}}}
\newcommand{\rrc}{\dot{\boldsymbol{\upsilon}}}
\newcommand{\B}{\mathbf{B}}
















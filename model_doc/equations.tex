\section{Governing Equations}

\subsection{Evolution of Species and Temperature}
A detailed chemical mechanism consists of a set of $\nrd$ reactions involving $\nsd$ species.  
Let $\xd = \left[x_{1}^{D}, \ldots, x_{\nsd}^{D}\right]^{\dagger} \in \mathbb{R}^{\nsd}$ denote 
a vector of species in the detailed model.  Then a set of coupled ODEs governing the evolution 
of each species and temperature is
\begin{align}
  \odeone{\xd}{t} &= \rrd \\
  \odeone{T}{t}   &= \frac{\displaystyle -\hd\lr{T}\cdot\odeone{\xd}{t} + \dot{\mathcal{Q}}}{\displaystyle \cpd\lr{T}\cdot\xd}
\end{align}
where $\dot{\mathcal{Q}}$ is a heating rate and $\cpd\lr{T}$ is the specific heat at constant 
pressure.  Note that
\begin{align}
  \cpd\lr{T} = \left.\pdeone{\hd}{T}\right|_{p}
\end{align}
where $\hd$ is the enthalpy of the detailed model.  In general, we denote the enthalpy by $h$ 
and the specific heat at constant pressure by $c_{p}$.
Two other useful relations are
\begin{align}
  h\lr{T} &= \int{c_{p}\lr{T} \ \mathrm{d}T} \label{eq:h_cp_rel} \\
  s\lr{T} &= \int{\frac{c_{p}}{T} \ \mathrm{d}T} \label{eq:s_cp_rel}
\end{align}
where $s\lr{T}$ is the entropy.

\subsection{Reaction Rates}
The reaction rate for species $k$ is given by 
\begin{align}
  \rrk{k} = \sum_{j=1}^{M}{\nu_{kj}r_{j}}
\end{align}
 where $\nu_{kj} = \nup{kj} - \nur{kj}$ is the difference between
 the stoichiometric coefficients for species $k$ 
in reaction $j$ and $r_{j}$ is the progress rate of reaction $j$.  
Note that $\nup{kj}$ and $\nur{kj}$ are the stoichiometric 
coefficients for the products and reactants respectively. The 
progress rate of reaction $j$ is described in the following 
subsections.
\subsubsection{Elementary Reactions}
For elementary reactions, the progress rate is given by 
\begin{align}
  r_{j} = \frate{j}\prod_{k=1}^{N}{\xk{k}^{\nur{kj}}} - 
          \rrate{j}\prod_{k=1}^{N}{\xk{k}^{\nup{kj}}}
\end{align}
where $\frate{j}$ and $\rrate{j}$ are the forward and reverse reaction rates 
respectively.  These are given by 
\begin{align}
  \frate{j} = A_{j}T^{\beta_{j}}\exp\lr{-E_{j}/RT}
\end{align}
and 
\begin{align}
  \rrate{j} = \frac{\frate{j}}{\Keq{j}}
\end{align}
where $\Keq{j}$ is the equilibrium constant.  It is computed from 
\begin{align}
  \Keq{j} = \lr{\frac{p_{a}}{RT}}^{\sum_{k=1}^{N}{\nu_{kj}}}\exp\lr{\frac{\Delta S_{j}}{R} - \frac{\Delta H}{RT}}
\end{align}
where $\Delta S_{j}$ and $\Delta H_{j}$ are the change in entropy and entropy 
between the products and reactants for reaction $j$.  These are usually given 
by the NASA polynomials.

\subsubsection{Unimolecular Reactions (Three-Body Reactions)}

\subsection{Comments on Reduced Kinetics}
In actual combustion simulations, we wish to reduce the size of the chemical kinetics system.  
To this end, let $\xr = \left[x_{1}^{R}, \ldots, x_{\nsr}^{R}\right]^{\dagger} \in \mathbb{R}^{\nsr}$ 
denote a vector of species in the reduced model.  Note that $\nsr < \nsd$.  Then, the reduced 
system of equations becomes,
\begin{align}
  \odeone{\xr}{t} &= \rrr \label{eq:reduced}\\
  \odeone{T}{t}   &= \frac{\displaystyle -\hr\lr{T}\cdot\odeone{\xr}{t} + \dot{\mathcal{Q}}}{\displaystyle \cpr\lr{T}\cdot\xr}. \label{eq:reduced_T}
\end{align}

Note that we can introduce a transition matrix, $\B = \left[\beta_{ij}\right]$, that 
redistributes neglected species from the detailed model to the reduced model.  That is
\begin{align}
  x_{i}^{R} = \sum_{j=\nsr + 1}^{\nsd}{\beta_{ij}x_{j}^{D}}. \label{eq:transition}
\end{align}
We will use this matrix in subsequent sections.

\section{Governing Equations}
We consider a perfectly stirred reactor at constant pressure but 
allow the volume to change.  The gas mixture in the chamber 
follows the ideal gas law 
\begin{align}
  pV = x R T \label{eq:ideal_gas_law}
\end{align}
where $p$ is the constant reactor pressure, $V = V\lr{t}$ is 
the volume of the reactor, $X$ represents the total amount 
of moles in the system, $R$ is the ideal gas constant and 
$T$ is the temperature of the mixture.  We consider a mixture 
of $N$ species undergoing $M$ reactions.  Hence, the total 
moles in the system is given by 
\begin{align}
  X = \sum_{k=1}^{N}{x_{k}}
\end{align}
where $x_{k}$ is the number of moles of species $k$.  We wish 
to determine the rate of change of the number of moles for 
each species.  We introduce the total molar concentration 
$\totmoleconc$, the molar fraction of species $k$, $\molefrac{k}$, 
and the molar concentration of species $k$, $\moleconc{k}$.  The 
rate of change of the molar concentration of species $k$ is 
then given by 
\begin{align}
  \odeone{\lr{\totmoleconc\molefrac{k}}}{\time} = \rrk{k}.
\end{align}
Integrating over the volumen $V\lr{t}$ and using the Leibnitz 
rule (a.k.a. the Reynolds transport theorem) we find 
\begin{align}
  \odeone{}{\time}\int_{V\lr{\time}}{\totmoleconc\molefrac{k} \ \mathrm{d}V} - 
     \int_{\partial V\lr{\time}}{\totmoleconc\molefrac{k}\mathbf{v}_{b}\cdot\mathbf{n} \ \mathrm{d}S}
     = 
     \rrk{k}V\lr{\time}
\end{align}
where $\mathbf{v}_{b}$ is the velocity of the boundary and $S$ is 
the bounding surface.  The second term on the left hand side is 
zero because there is no surface flux in our system.  Hence 
\begin{align}
  \odeone{\lr{\totmoleconc\molefrac{k}V}}{\time} = 
     \rrk{k}V\lr{\time}.
\end{align}
Of course $\totmoleconc\molefrac{k} = \moleconc{k}$.  Therefore, 
after expanding the time derivative, we obtain 
\begin{align}
  \odeone{\moleconc{k}}{\time} = \rrk{k} - 
    \moleconc{k}\frac{1}{V\lr{\time}}\odeone{V}{\time}
  \label{eq:mole_conc_evo_V}.
\end{align}
Equation~\eqref{eq:mole_conc_evo_V} can be derived starting 
from change in total system mass and then converting to 
molar units.  However, since the present work is done in terms 
of moles, we elected to work with moles from the outset.

The energy equation can also be derived 
(\dls{details not shown here yet}).  It turns out to be 
\begin{align}
  \odeone{T}{\time} = \frac{\displaystyle 
     -\sum_{k=1}^{N}{h_{k}W_{k}\rrk{k}} + \dfrac{\heatrate}{V}}
  {\displaystyle \sum_{k=1}^{N}{c_{p,k}W_{k}\moleconc{k}}}
\end{align}
where $\heatrate$ is a user-provided heating rate.

The species concentration equation can be rearranged 
considerably.  Primarily, the volume time-derivative 
must be dealt with.  To do so, we consider the ideal gas 
law~\eqref{eq:ideal_gas_law}.  Solving for the volume we 
have 
\begin{align}
  V\lr{\time} = \frac{X R T}{p}.
\end{align}
The time-derivative is 
\begin{align}
  \odeone{V}{\time} = \frac{X R}{p}\odeone{T}{\time}.
  \label{eq:dVdt}
\end{align}
It follows that 
\begin{align}
  \frac{1}{V}\odeone{V}{\time} = \frac{R}{p}\odeone{T}{\time} 
       \sum_{k=1}^{N}{\moleconc{k}}
\end{align}
or in another way (using~\eqref{eq:ideal_gas_law}) 
\begin{align}
  \frac{1}{V}\odeone{V}{\time} = \frac{1}{T}\odeone{T}{\time}.
\end{align}
Using the volume time-derivative in~\eqref{eq:mole_conc_evo_V} 
we obtain 
\begin{align}
  \odeone{\moleconc{k}}{\time} = \rrk{k} - 
    \frac{1}{T}\odeone{T}{\time}\moleconc{k}
  \label{eq:mole_conc_evo}.
\end{align}

\subsection{Some Additional Notes}
In deriving the above evolution equations, we worked in 
molar units.  However, most combustion people seem to 
think in terms of mass units.  Hence, the ideal gas law 
is typically written as 
\begin{align}
  p = \rho \frac{R}{W} T
\end{align}
where $W$ is the mean molecular weight of the mixture, 
$\rho = m / V$ is the density of the mixture and $m$ is 
the total mass of the mixture.  The volume is then given 
by 
\begin{align}
  V = \frac{m R T}{W p}
\end{align}
and the time-derivative of volume is 
\begin{align}
  \odeone{V}{\time} = \frac{m R}{p} 
    \left[\frac{1}{W}\odeone{T}{\time} - 
    \frac{T}{W^{2}}\odeone{W}{\time}\right].
  \label{eq:dVdt_mass}
\end{align}
If the mean molecular weight was constant in time, 
then~\eqref{eq:dVdt_mass} would be similar to~\eqref{eq:dVdt}. 
To see that the mean molecular weight is indeed constant 
in time we consider the definition of $W$ in terms of 
the mole fraction, 
\begin{align}
  W = \sum_{k=1}^{N}{W_{k}\molefrac{k}}.
\end{align}
Then 
\begin{align}
  \odeone{W}{\time} &= 
    \sum_{k=1}^{N}{W_{k}\odeone{\molefrac{k}}{\time}} \\
    &= \sum_{k=1}^{N}{\frac{W_{k}}{X}\odeone{x_{k}}{\time}} \\
    &= \frac{1}{X}\sum_{k=1}^{N}{W_{k}\odeone{}{\time}\lr{V \moleconc{k}}} \\
    &= \frac{V}{X}\sum_{k=1}^{N}{W_{k}\rrk{k}} \\
    &= 0.
\end{align}
The last line is possible due to conservation of mass.  
Conservation of mass is expressed as 
\begin{align}
  \sum_{k=1}^{N}{W_{k}\nu_{kj}} = 0, \qquad \forall \ j.
\end{align}
Thus,
\begin{align}
  \sum_{k=1}^{N}{W_{k}\rrk{k}} = 
    \sum_{j=1}^{M}{q_{j}\lr{\sum_{k=1}^{N}W_{k}\nu_{kj}}} = 0.
\end{align}
Hence,~\eqref{eq:dVdt_mass} becomes 
\begin{align}
  \odeone{V}{\time} = \frac{m}{W}\frac{R}{p}\odeone{T}{\time}.
  \label{eq:dVdt_mass_Wconst}
\end{align}
We can write~\eqref{eq:dVdt_mass_Wconst} in terms of the total 
moles to recover~\eqref{eq:dVdt} using the relation $m = X W$. 
To see that this relation is true we have 
\begin{align}
  m &= \sum_{k=1}^{N}{mY_{k}} \\
    &= \sum_{k=1}^{N}{m\frac{W_{k}}{\rho}\moleconc{k}} \\
    &= \sum_{k=1}^{N}{W_{k}V\moleconc{k}} \\
    &= \sum_{k=1}^{N}{W_{k}x_{k}} \\
    &= X\sum_{k=1}^{N}{W_{k}\molefrac{k}} \\
    &= X W.
\end{align}



Two other useful relations are
\begin{align}
  h\lr{T} &= \int{c_{p}\lr{T} \ \mathrm{d}T} \label{eq:h_cp_rel} \\
  s\lr{T} &= \int{\frac{c_{p}}{T} \ \mathrm{d}T} \label{eq:s_cp_rel}
\end{align}
where $s\lr{T}$ is the entropy.



%\subsection{Evolution of Species and Temperature}
%A detailed chemical mechanism consists of a set of $\nrd$ reactions involving $\nsd$ species.  
%Let $\xd = \left[x_{1}^{D}, \ldots, x_{\nsd}^{D}\right]^{\dagger} \in \mathbb{R}^{\nsd}$ denote 
%a vector of species in the detailed model.  Then a set of coupled ODEs governing the evolution 
%of each species and temperature is
%\begin{align}
%  \odeone{\xd}{t} &= \rrd \\
%  \odeone{T}{t}   &= \frac{\displaystyle -\hd\lr{T}\cdot\odeone{\xd}{t} + \dot{\mathcal{Q}}}{\displaystyle \cpd\lr{T}\cdot\xd}
%\end{align}
%where $\dot{\mathcal{Q}}$ is a heating rate and $\cpd\lr{T}$ is the specific heat at constant 
%pressure.  Note that
%\begin{align}
%  \cpd\lr{T} = \left.\pdeone{\hd}{T}\right|_{p}
%\end{align}
%where $\hd$ is the enthalpy of the detailed model.  In general, we denote the enthalpy by $h$ 
%and the specific heat at constant pressure by $c_{p}$.
%Two other useful relations are
%\begin{align}
%  h\lr{T} &= \int{c_{p}\lr{T} \ \mathrm{d}T} \label{eq:h_cp_rel} \\
%  s\lr{T} &= \int{\frac{c_{p}}{T} \ \mathrm{d}T} \label{eq:s_cp_rel}
%\end{align}
%where $s\lr{T}$ is the entropy.

\subsection{Reaction Rates}
The reaction rate for species $k$ is given by 
\begin{align}
  \rrk{k} = \sum_{j=1}^{M}{\nu_{kj}r_{j}}
\end{align}
 where $\nu_{kj} = \nup{kj} - \nur{kj}$ is the difference between
 the stoichiometric coefficients for species $k$ 
in reaction $j$ and $r_{j}$ is the progress rate of reaction $j$.  
Note that $\nup{kj}$ and $\nur{kj}$ are the stoichiometric 
coefficients for the products and reactants respectively. The 
progress rate of reaction $j$ is described in the following 
subsections.
\subsubsection{Elementary Reactions}
For elementary reactions, the progress rate is given by 
\begin{align}
  r_{j} = \frate{j}\prod_{k=1}^{N}{\xk{k}^{\nur{kj}}} - 
          \rrate{j}\prod_{k=1}^{N}{\xk{k}^{\nup{kj}}} \label{eq:rj_elem}
\end{align}
where $\frate{j}$ and $\rrate{j}$ are the forward and reverse reaction rates 
respectively.  These are given by 
\begin{align}
  \frate{j} = A_{j}T^{\beta_{j}}\exp\lr{-E_{j}/RT}
\end{align}
and 
\begin{align}
  \rrate{j} = \frac{\frate{j}}{\Keq{j}}
\end{align}
where $\Keq{j}$ is the equilibrium constant.  It is computed from 
\begin{align}
  \Keq{j} = \lr{\frac{p_{a}}{RT}}^{\sum_{k=1}^{N}{\nu_{kj}}}\exp\lr{\frac{\Delta S_{j}}{R} - \frac{\Delta H}{RT}}
\end{align}
where $\Delta S_{j}$ and $\Delta H_{j}$ are the change in entropy and entropy 
between the products and reactants for reaction $j$.  These are usually given 
by the NASA polynomials.

\subsubsection{Unimolecular Reactions (Three-Body Reactions)}
Unimolecular reactions involve the dissociation of a single 
homogeneous species (e.g. \ce{O2}) into its component elements 
with the aid of the other species in the mixture (the third body). 
An example of such a reaction is 
\begin{align}
  \ce{O + O + M <=> O2 + M}.
\end{align}
The reaction progress rate~\eqref{eq:rj_elem} is modified to be 
\begin{align}
  r_{j}^{u} = \xk{M}r_{j}
\end{align} 
where the concentration of the third body $\xk{M}$ is given by 
\begin{align}
  \xk{M} = \sum_{k=1}^{N}{\epsilon_{kj}\xk{k}}
\end{align}
and $\epsilon_{kj}$ is the efficiency of species $k$ in the 
unimolecular reaction $j$.  The forward and reverse reaction 
rate coefficients are the same as for the elementary reactions.

\subsubsection{Unimolecular Reactions (Three-Body Falloff Reactions)}
Sometimes it is possible for the reaction rate coefficients to be 
have a pressure dependency.  These are called falloff reactions 
because of the way that the reaction rate ``falls off'' from a 
constant at low pressures.  The forward reaction rate coefficient 
is given by 
\begin{align}
  \frate{} = \lr{\frac{1}{k_{0}\xk{M}} + \frac{1}{k_{\infty}}}^{-1}F
\end{align}
where $k_{0}$ and $k_{\infty}$ are the forward reaction rate 
coefficients at low and high pressures, respectively.  Note that 
the pressure dependence is actually implicity represented in $\xk{M}$ 
since $\xk{M}$ and the pressure are related linearly.  $F$ is 
a fudge factor.  For Lindemann falloffs $F = 1$. 
However, there are many other detailed treatments of $F$.  One 
such representation is called the Troe falloff.  Brace yourself, 
this one is a bit of a mess.  The fudge factor is 
given by
\begin{align}
  \log_{10}\lr{F} = \frac{\displaystyle \log_{10}\lr{F_{c}\lr{T}}}
    {\displaystyle 1 + \lr{\frac{\displaystyle\log_{10}\lr{\frac{k_{0}\xk{M}}{k_{\infty}}} + c}{\displaystyle N - d\lr{\log_{10}\lr{\frac{k_{0}\xk{M}}{k_{\infty}}} + c}}}^{2}}
\end{align}
where 
\begin{align}
  c &= -0.4 - 0.67\log_{10}\lr{F_{c}\lr{T}} \\
  N &= 0.75 - 1.27\log_{10}\lr{F_{c}\lr{T}} \\
  d &= 0.14
\end{align}
are emperical parameters that depend on the central broadening factor which 
is in turn given by 
\begin{align}
  F_{c}\lr{T} = \lr{1-a}\exp\lr{-\frac{T}{T_{3}}} + 
                      a \exp\lr{-\frac{T}{T_{1}}} + 
                        \exp\lr{-\frac{T_{2}}{T}}.
\end{align}
The parameters $a$, $T_{1}$, $T_{2}$, and $T_{3}$ are specified by the 
user.

\subsection{Comments on Reduced Kinetics}
In actual combustion simulations, we wish to reduce the size of the chemical kinetics system.  
To this end, let $\xr = \left[x_{1}^{R}, \ldots, x_{\nsr}^{R}\right]^{\dagger} \in \mathbb{R}^{\nsr}$ 
denote a vector of species in the reduced model.  Note that $\nsr < \nsd$.  Then, the reduced 
system of equations becomes,
\begin{align}
  \odeone{\xr}{t} &= \rrr \label{eq:reduced}\\
  \odeone{T}{t}   &= \frac{\displaystyle -\hr\lr{T}\cdot\odeone{\xr}{t} + \dot{\mathcal{Q}}}{\displaystyle \cpr\lr{T}\cdot\xr}. \label{eq:reduced_T}
\end{align}

Note that we can introduce a transition matrix, $\B = \left[\beta_{ij}\right]$, that 
redistributes neglected species from the detailed model to the reduced model.  That is
\begin{align}
  x_{i}^{R} = \sum_{j=\nsr + 1}^{\nsd}{\beta_{ij}x_{j}^{D}}. \label{eq:transition}
\end{align}
We will use this matrix in subsequent sections.
















\subsection{State Perturbation}
In the previous formulations, we have focused on directly
perturbing the right-hand-side of the reduced model.  This 
has proven very challenging.  In particular, it is quite 
difficult to create a general perturbation that respects 
all the constraints.  The most challenging of these 
appears to be satisfaction of non-negativity.

Another viewpoint that may prove useful is to perturb the 
actual state at the discrete level.  For example, suppose 
we do a simple forward Euler temporal discretization.  
Then, 
\begin{align}
  \x^{\lr{n+1}} = \x^{\lr{n}} + \Delta t \mathbf{f}\lr{\x^{\lr{n}}, T^{\lr{n}}} 
       + \xi.
\end{align}
The basic idea is that the state that results from using 
the reduced model is a perturbation of the state 
that would result from the detailed model.  If we happened 
to choose 
\begin{align}
  \xi \sim \mathcal{N}\lr{0,\Delta t}
\end{align}
then the perturbation would an induce a right-hand-side 
perturbation in the form, 
\begin{align}
  \odeone{\x}{t} = \mathbf{f}\lr{\x, T} + \odeone{W}{t}
  \label{eq:stoch_chem}
\end{align}
where $W$ is the stochastic Wiener process. Such a choice is 
really just one simple example of a type of perturbation.  
We would really like to introduce a perturbation that is 
consistent with the way in which the detailed model is 
reduced to the reduced model.  For example, if the model 
reduction occurs by neglecting fast time-scales, then 
errors are introduced through that reduction in accordance 
with the significance of ignoring each species or reaction.  
The choice of the state-perturbation may be guided by 
considering actual reduction strategies in the literature.

Many reduction strategies have been introduced throughout 
the years.  Early reduction strategies include the 
quasi-steady-state approximation and the partial-equilibrium 
approximation.  More sophisticated methods include the 
directed graph relation (DAG) method, computational 
singular perturbation (CSP) -based methods and level of 
importance (LOI) analysis.  Many mechanism-reduction 
strategies do not rigorously consider effects of neglecting 
certain time-scales.  The CSP makes an attempt at 
rigorously understanding which reactions and species are 
involved in processes that occur at very fast time-scales.  
The CSP relies on a paramter $\epsilon_{M}$ which is a 
ratio between the slowest fast time scale and the 
fastest slow time scale.  If this parameter is small then 
the CSP has successfully separated the fast and slow 
time scales.  If we consider $\epsilon_{M}$ as an 
indication of the error in reduced models generated from 
the CSP, then it may be interesting to choose the 
state perturbation $\xi$ such that its variance is 
proportional to this parameter.  In that way, when 
$\epsilon_{M}$ is small the variance of $\xi$ can be 
small since the reduced model is good whereas when 
$\epsilon_{M}$ is large the variance of $\xi$ can be 
large since the reduced model is probably not very 
good.

Interestingly, the form of the perturbed ODE presented in~\eqref{eq:stoch_chem} 
is similar in form to that derived from the stochastic simulation algorithm.  
Assuming the reactant populations are sufficiently large and that the 
right-hand-side function does not change too much over a time interval 
$\left[\left.t, t + \Delta t\right)\right.$ then the system satisfies 
\begin{align}
  x_{k}\lr{t + \Delta t} = x_{k} + \Delta t \sum_{j=1}^{M}{\nu_{kj}q_{j}V} + 
        \sum_{j=1}^{M}{\nu_{kj}\sqrt{q_{j}}\mathcal{N}\lr{0,1}\sqrt{\Delta t}}.
\end{align}



\section{Inadequacy Formulations}
Equations~\eqref{eq:reduced} and~\eqref{eq:reduced_T} are not always sufficient to represent 
chemical reactions with the fidelity needed in combustion applications.  We introduce an 
additional term to the reduced model to capture the effects of the neglected reactions and 
species.  The general form is 
\begin{align}
  \odeone{\xs}{t} &= \rrr\lr{\xr, T} + \rrs\lr{\xr, T; \pars}
\end{align}
and we wish to develop a reasonable form for $\rrs\lr{\xr, T; \pars}$.  Note that $\pars$ 
represents stochastic parameters that enter into the model.

\subsection{Initial Formulation}
This initial formulation is due to Rebecca's thesis work.  We augment the species in the 
reduced model by $\na$ ``catchall'' species which are intended to mimic the effect of the 
species that were neglected in going from the detailed to the reduced model.  Catchall 
species and their quantities are denoted with a supersript $\prime$.  The species 
concentrations are now given by 
$\x = \left[x_{1}, \ldots, \x_{\nsr}, \x_{\nsr+1}, \ldots, \x_{\nsr + \na}\right]^{\dagger}$ 
where $\na$ is the number of atoms in the system.  The enthalpies are augmented similarly.  
We therefore introduce an enhanced reduced model of the form,
\begin{align}
  \odeone{\x}{t} &= \rr + \left[\Sop \x + \rrc\right] \label{eq:inad_reb} \\
  \odeone{T}{t}  &= \frac{-\displaystyle \h\lr{T}\cdot\odeone{\x}{t} + \dot{\mathcal{Q}}}
                         {\displaystyle \cp\lr{T}\cdot\x}. \label{eq:energy_reb}
\end{align}
The operators $\Sop$ and $\rrc$ are stochastic in nature and have several parameters that 
must be calibrated.  We must also specify a form for the enthalpies of the ``catchall'' 
species.  This was done by assuming a simple linear temperature dependence of the specific 
heat on the temperature.  In general,  
\begin{align}
  \hic{k} = \sum_{i=0}^{p}{\alpha_{ki}T^{i}}
\end{align}
and 
\begin{align}
  \cpic{k} = \sum_{i=1}^{p}{i\alpha_{ki}T^{i-1}}
\end{align}
where $p-1$ is the polynomial order being used to represent the specific heat.  Note that 
using~\eqref{eq:s_cp_rel} we have the additional relationship 
\begin{align}
  \sic{k} = \gamma_{k0} + \alpha_{k1}\ln\lr{T} + \sum_{i=2}^{p}{\frac{i}{i-1}\alpha_{ki}T^{i-1}}.
\end{align}
When $p=2$ we have simply 
\begin{align}
  \hic{k}  &= \alpha_{k0} + \alpha_{k1}T + \alpha_{k2}T^{2} \\ 
  \cpic{k} &= \alpha_{k1} + 2\alpha_{k2}T \\
  \sic{k}  &= \gamma_{k0}  + \alpha_{k1}\ln\lr{T} + 2\alpha_{k2}T.
\end{align}



\subsubsection{Global Temperature Dependence}
The inadequacy portion of the model is contained in a linear part $\Sop$ and a nonlinear part 
$\rrc$ which are both temperature independent.  Temperature dependence is built in through 
a prefator $g\lr{T}$.  
\begin{align}
  \odeone{\x}{t} &= \rr + g\lr{T}\left[\Sop \x + \rrc\right] \\
  \odeone{T}{t}  &= \frac{-\displaystyle \h\lr{T}\cdot\odeone{\x}{t} + \dot{\mathcal{Q}}}{\displaystyle \cp\lr{T}\cdot\x}.
\end{align}
Two forms for the prefactor are the Arrhenius form and the hyperbolic tangent switch. 
The Arrhenius form is
\begin{align}
  g_{A}\lr{T} = \exp\lr{-T_{ag} / T}
\end{align}
where $T_{ag}$ is the global activation temperature.  The global hyperbolic switch has 
the form
\begin{align}
  g_{S}\lr{T} = \tanh\lr{T-T_{ig}} = \tanh\lr{T - T_{adg}}
\end{align}
where $T_{ig}$ and $T_{adg}$ are the global ignition temperature and the global 
adiabatic flame temperature.

\subsection{Dissociation-Recombination Reactions}
The linear part of the original inadequacy formulation~\ref{eq:inad_reb} is very 
high-dimensional.  We would like to reduce the dimensionality of the inadequacy 
formulation.  This will provide some additional flexibility when introducing a 
temperature dependence (e.g. we may be able to get with something more fine-grained 
than a global temperature dependence).

As a first step, we can introduce a set of reversible dissociation-recombination 
(D-R) reactions.  These reactions will allow species to dissociate into elemental 
catchall species.  For example, the \ce{H2}-\ce{O2} system consists of the set 
of species $\left\{\ce{H}, \ce{O}, \ce{H2}, \ce{O2}, \ce{OH}, \ce{H2O}, \ce{HO2}\right\}$. 
The set of catchall species is $\left\{\ce{H}^{\prime}, \ce{O}^{\prime}\right\}$. 
An example of a D-R reaction would be 
\begin{align}
  \ce{H2O <=> 2H^{\prime} + O^{\prime}}.
\end{align} 
We would then form the ODE system as usual and include this new reaction.  We 
would of course also need to provide a form for the enthalpy and a form for 
entropy for the catchall species since we are now dealing with a reversible 
reaction.  

These worked amazingly at first but ultimately led to numerical problems.

\subsection{Transition Matrix Formulaton}
We are once again re-thinking the inadequacy formulation.  The inadequacy 
formulation takes the usual form,
\begin{align}
  \odeone{\mathbf{x}}{\time} = \mathbf{f}\lr{\mathbf{x}, T} + \mathbf{h}\lr{\mathbf{x}, T}.
\end{align}
We consider the inadequacy model $\mathbf{h}$ in the form,
\begin{align}
  \mathbf{h}\lr{\mathbf{x}, T} = A\widetilde{\mathbf{x}} + N_{A}\widehat{\mathbf{x}}.
  \label{eq:trans_inad}
\end{align}
In~\eqref{eq:trans_iand}, $A$ is the species to atoms transition matrix and 
$N_{A}$ is a matrix whose columns span the nullspace of $A$.  The quantities 
$\widetilde{\mathbf{x}}$ and $\widehat{\mathbf{x}}$ are random variables. 
Note that we must find a way to build the temperature dependence into 
these perturbations.  The transition matrix $A$ has size $\na\times\ns$. 
For the \ce{H2}/\ce{O2} system, using the reduced species set, the 
transition matrix is
\begin{align}
  A = 
  \begin{bmatrix}
    2 & 0 & 1 & 0 & 1 & 1 & 2 \\
    0 & 2 & 0 & 1 & 1 & 2 & 1
  \end{bmatrix}.
\end{align}
Note that the total number of atoms of each time is given by 
\begin{align}
  \mathbf{a}_{\textrm{tot}} = A\mathbf{x}.
\end{align}
We know that the total number of atoms is conserved.
Thus
\begin{align}
  0 = \odeone{\mathbf{a}_{\textrm{tot}}}{\time} = 
  A \odeone{\mathbf{x}}{\time} = A\rr.
\end{align}
Solutions to $A\mathbf{x} = 0$ are given by 
\begin{align}
  \mathbf{x} = 
  \begin{bmatrix} -1/2 \\ 0 \\ 1 \\ 0 \\ 0 \\ 0 \\ 0 \end{bmatrix}x_{3} 
  + 
  \begin{bmatrix} 0 \\ 1/2 \\ 0 \\ 1 \\ 0 \\ 0 \\ 0 \end{bmatrix}x_{4} 
  + 
  \begin{bmatrix} -1/2 \\ \displaystyle -1/2 \\ 0 \\ 0 \\ 1 \\ 0 \\ 0 \end{bmatrix}x_{5} 
  + 
  \begin{bmatrix} -1/2 \\ -1 \\ 0 \\ 0 \\ 0 \\ 1 \\ 0 \end{bmatrix}x_{6} 
  + 
  \begin{bmatrix} -1 \\ -1/2 \\ 1 \\ 0 \\ 0 \\ 0 \\ 1 \end{bmatrix}x_{7}.
\end{align}
The dimension of the null space is therefore $5$.  It may be that in general 
the dimension of the nullspace of $A$ is $\ns - \na$ but I don't have a 
proof of this at this exact moment.

The form of~\eqref{eq:trans_inad} is such that we are perturbing the 
reduced model in the direction of range of $A$ as well as the direction 
of the null space of $A$.  The first perturbation (in the direction 
of the range of $A$) may result in violations of conservation of atoms.

We now consider conservation of atoms.  It may be necessary to perturb
the total atoms to guarantee that atoms are conserved.  We introduce 
the virutal species $\xv$ to help enforce conservation.  Thus, 
\begin{align}
  \mathbf{x}_{\textrm{tot}} = A\mathbf{x} + \xv
\end{align}
where $\xv$ is a vector containing the virtual species and acting 
as a correction to guarantee conservation.  Now we have 
\begin{align}
  0 = \odeone{\mathbf{x}_{\textrm{tot}}}{\time} &= A\odeone{\mathbf{x}}{\time} + 
       \odeone{\xv}{\time} \\
    &= A\lr{\mathbf{f} + \mathbf{h}} + \odeone{\xv}{\time}.
\end{align}
Thus, the virtual species must evolve according to 
\begin{align}
  \odeone{\xv}{\time} = -A\lr{\mathbf{f} + \mathbf{h}}.
\end{align}

\begin{tcolorbox}
  WARNING! The dimensions of this formulation don't even
           make sense.  Need to rethink everything.
\end{tcolorbox}

\section{Basic Chemistry Notes}

Within the mixture, assume that each species obeys the 
ideal gas law so that 
\begin{align}
  p_{k} = \frac{\rho_{k}RT}{W_{k}}
\end{align}
where $p_{k}$ is the pressure of speices $k$, 
$\rho_{k}$ is the density of species $k$, $W_{k}$ 
is the molecular weight of species $k$, $R$ is the 
universal gas contant and $T$ is the temperature of the 
mixture.  The thermodynamic pressure of the mixture is 
\begin{align*}
  p &= \sum_{k=1}^{N}{p_{k}} \\
    &= \sum_{k=1}^{N}{\frac{\rho_{k}}{W_{k}}}RT \\
    &= \sum_{k=1}^{N}{\frac{\rho Y_{k}}{W_{k}}RT} \\
    &= \rho R T \sum_{k=1}^{N}{\frac{Y_{k}}{W_{k}}} \\
    &= \frac{\rho R T}{W}
\end{align*}
where the relations 
\begin{align}
  \rho_{k} = \rho Y_{k} \qquad \text{and} \qquad 
  \frac{1}{W} = \sum_{k=1}^{N}{\frac{Y_{k}}{W_{k}}}
\end{align}
where used.  Note that $Y_{k}$ denotes the mass 
fraction of species $k$.

We can play additional games with the ideal gas 
relationship.  These games will shed light on 
some annoying nomenclature that arises frequently 
when considering three-body reactions.  Note that 
\begin{align}
  \frac{\rho}{W} &= \sum_{k=1}^{N}{\frac{\rho Y_{k}}{W_{k}}} \\
                 &= \sum_{k=1}^{N}{\left[X_{k}\right]}
\end{align}
where $X_{k} = \dfrac{\rho Y_{k}}{W_{k}}$ is the molar 
concentration of species $k$.  The ideal gas law therefore 
becomes 
\begin{align}
  p = \sum_{k=1}^{N}{\left[X_{k}\right]}RT.
\end{align}
Note that the thermodynamic pressure is directly 
proportional to the molar concentrations.  In 
three-body reactions, the concentration of the 
``third body'', \ce{M} is usually written as a linear 
combination of each species in the mixture.  That is 
\begin{align}
  \left[\ce{M}\right] = \sum_{k=1}^{N}{\left[X_{k}\right]}.
\end{align}
Hence, when the literature states that it is equivalent 
to consider the concentration of the third body or 
pressure this is what they mean.


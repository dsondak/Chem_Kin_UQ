\documentclass[fontsize=12pt, % Document font size
               paper=a4, % Document paper type
               hyperref]{report}

\usepackage[bottom=10em]{geometry} % Reduces the whitespace at the bottom of the page so more text can fit

\usepackage[english]{babel} % English language
\usepackage{lipsum} % Used for inserting dummy 'Lorem ipsum' text into the template

\usepackage[utf8]{inputenc} % Uses the utf8 input encoding
\usepackage[T1]{fontenc} % Use 8-bit encoding that has 256 glyphs

\usepackage[osf]{mathpazo} % Palatino as the main font
\linespread{1.05}\selectfont % Palatino needs some extra spacing, here 5% extra
\usepackage[scaled=.88]{beramono} % Bera-Monospace
\usepackage[scaled=.86]{berasans} % Bera Sans-Serif

\usepackage{booktabs,array} % Packages for tables

\usepackage{amsmath} % For typesetting math
\usepackage{graphicx} % Required for including images
\usepackage{etoolbox}
\usepackage[norule]{footmisc} % Removes the horizontal rule from footnotes
\usepackage{lastpage} % Counts the number of pages of the document

\usepackage{natbib}

\usepackage[dvipsnames]{xcolor}  % Allows the definition of hex colors
\definecolor{titleblue}{rgb}{0.16,0.24,0.64} % Custom color for the title on the title page
\definecolor{linkcolor}{rgb}{0,0,0.42} % Custom color for links - dark blue at the moment

\DeclareFixedFont{\textcap}{T1}{phv}{bx}{n}{1.5cm} % Font for main title: Helvetica 1.5 cm
\DeclareFixedFont{\textaut}{T1}{phv}{bx}{n}{0.8cm} % Font for author name: Helvetica 0.8 cm

\usepackage[nouppercase,headsepline]{scrpage2} % Provides headers and footers configuration
\pagestyle{scrheadings} % Print the headers and footers on all pages
\clearscrheadfoot % Clean old definitions if they exist

\automark[chapter]{chapter}
\ohead{\headmark} % Prints outer header

\setlength{\headheight}{25pt} % Makes the header take up a bit of extra space for aesthetics
\setheadsepline{.4pt} % Creates a thin rule under the header

\ofoot[\normalfont\normalcolor{\thepage\ |\  \pageref{LastPage}}]{\normalfont\normalcolor{\thepage\ |\  \pageref{LastPage}}} % Creates an outer footer of: "current page | total pages"

% For color boxes
\usepackage{tcolorbox}

% For chemistry
\usepackage{mhchem}

% Hyperlink configuration
\usepackage[
    pdfauthor={}, % Your name for the author field in the PDF
    pdftitle={Laboratory Journal}, % PDF title
    pdfsubject={}, % PDF subject
    bookmarksopen=true,
    linktocpage=true,
    urlcolor=linkcolor, % Color of URLs
    citecolor=linkcolor, % Color of citations
    linkcolor=linkcolor, % Color of links to other pages/figures
    backref=page,
    pdfpagelabels=true,
    plainpages=false,
    colorlinks=true, % Turn off all coloring by changing this to false
    bookmarks=true,
    pdfview=FitB]{hyperref}

\usepackage[stretch=10]{microtype} % Slightly tweak font spacing for aesthetics

%\setlength\parindent{0pt} % Uncomment to remove all indentation from paragraphs

%----------------------------------------------------------------------------------------
%       NEW COMMANDS
%----------------------------------------------------------------------------------------

\newcommand{\lr}[1]{\left(#1\right)}
\newcommand{\pdeone}[2]{\frac{\partial #1}{\partial #2}}
\newcommand{\pden}[3]{\frac{\partial^{#3} #1}{\partial #2^{#3}}}
\newcommand{\odeone}[2]{\frac{\mathrm{d} #1}{\mathrm{d} #2}}
\newcommand{\nup}[1]{\nu_{#1}^{\prime\prime}}
\newcommand{\nur}[1]{\nu_{#1}^{\prime}}
\newcommand{\frate}[1]{K_{#1}^{\left(f\right)}}
\newcommand{\rrate}[1]{K_{#1}^{\left(r\right)}}

\newcommand{\nrd}{M_{D}}
\newcommand{\nsd}{N_{D}}
\newcommand{\xd}{\mathbf{x}^{D}}
\newcommand{\nrr}{M_{R}}
\newcommand{\nsr}{N_{R}}
\newcommand{\na}{N_{a}}
\newcommand{\xr}{\mathbf{x}^{R}}
\newcommand{\x}{\mathbf{x}}

\newcommand{\rrd}{\dot{\boldsymbol{\omega}}^{D}}
\newcommand{\rrr}{\dot{\boldsymbol{\omega}}^{R}}
\newcommand{\rr}{\dot{\boldsymbol{\omega}}}
\newcommand{\hd}{\mathbf{h}^{D}}
\newcommand{\hr}{\mathbf{h}^{R}}
\newcommand{\h}{\mathbf{h}}
\newcommand{\hc}{\mathbf{h}^{\prime}}
\newcommand{\cpd}{\mathbf{c}_{p}^{D}}
\newcommand{\cpr}{\mathbf{c}_{p}^{R}}
\newcommand{\cp}{\mathbf{c}_{p}}

\newcommand{\Sop}{\boldsymbol{\mathcal{S}}}
\newcommand{\rrc}{\dot{\boldsymbol{\upsilon}}}

















\begin{document}

\title{\textcolor{titleblue}{Kinetics Formulation} \\[1cm]}

\author{David Sondak}
\date{} % No date by default, add \today if you wish to include the publication date

\maketitle % Title page

\newpage % Start lab look on a new page

\pagestyle{scrheadings} % Begin using headers

\section{Kinetics Formulation}
We consider a perfectly stirred reactor at constant pressure but 
allow the volume to change.  The gas mixture in the chamber 
is assumed to follow the ideal gas law 
\begin{align}
  pV = X R T \label{eq:ideal_gas_law}
\end{align}
where $p$ is the constant reactor pressure, $V = V\lr{t}$ is 
the volume of the reactor, $X$ represents the total moles in 
the mixture, $R$ is the ideal gas constant and $T$ is the 
temperature of the mixture.  We consider a mixture of $N$ 
species undergoing $M$ reactions.  Note that the total 
moles in the mixture is simply the sum of the moles of each 
species, 
\begin{align}
  X = \sum_{k=1}^{N}{\moles{k}}.
\end{align}

The total moles of each species evolves according to 
\begin{align}
  \odeone{\moles{k}}{\time} = \rrk{k}V.
\end{align}
The energy equation can be written as 
\begin{align}
  \odeone{T}{\time} = \frac{-\h\cdot\rrk{k} V + \heatrate}{\cp\cdot\x}
\end{align}
where $\h$ is the enthalpy in molar units, $\cp$ is the specific 
heat at constant pressure (also in molar units), $\heatrate$ 
is the user-supplied heating rate (a constant in what follows), 
and $\x$ is the number of moles of each species. 

\subsection{Jacobian}
Before presenting the Jacobian, we introduce two quantities 
that will appear in the Jacobian.  These are 
\begin{align}
  \pdeone{V}{x_{j}} = \frac{R T}{p}\mathbf{1}_{j}, \qquad 
  \pdeone{V}{T} = \frac{R X}{p}.
\end{align}
where $\mathbf{1}_{j}$ is a row vector of $N$ ones.

The system of equations can be written as 
\begin{align}
  \begin{bmatrix}
    \displaystyle \odeone{\x}{\time} \\[1.0em] \displaystyle \odeone{T}{\time} 
  \end{bmatrix}
  = 
  \begin{bmatrix}
    \mathbf{f} \\ f_{_{T}}.
  \end{bmatrix}
\end{align}

The Jacobian matrix consists of four blocks.  We write 
\begin{align}
  \mathbf{J} = 
  \begin{bmatrix}
    \displaystyle \pdeone{\mathbf{f}}{\x} & \displaystyle \pdeone{\mathbf{f}}{T} \\[1.0em]
    \displaystyle \pdeone{f_{_{T}}}{\x} & \displaystyle \pdeone{f_{_{T}}}{T}
  \end{bmatrix}.
\end{align}
The upper $N \times N + 1$ block of the Jacobian is given by  
\begin{align}
  \pdeone{f_{k}}{x_{j}} &= \pdeone{\rrk{k}}{x_{j}}V + \rrk{k}\pdeone{V}{x_{j}} \\
  \pdeone{f_{k}}{T} &= \pdeone{\rrk{k}}{T}V + \rrk{k}\pdeone{V}{T}.
\end{align}
The lower $1 \times N + 1$ block of the Jacobian is given by 
\begin{align}
  \pdeone{f_{_{T}}}{x_{j}} &= 
    \frac{\displaystyle -\lr{\cp\cdot\x}\left[\h\cdot\mathbf{J}_{\omega}\mathbf{e}_{j}V + 
           \lr{\h\cdot\rr}\pdeone{V}{x_{j}}\right] - \left[-\h\cdot\rr V + 
           \heatrate\right]\lr{\cp\cdot\mathbf{e}_{j}}}{\displaystyle \lr{\cp\cdot\x}^{2}} \\
  \pdeone{f_{_{T}}}{T} &= 
  \frac{\displaystyle -\lr{\cp\cdot\x}\left[\h\cdot\lr{\rr\pdeone{V}{T} + 
             \pdeone{\rr}{T}V} + \cp\cdot\rr V\right] - 
        \left[-\h\cdot\rr V + \heatrate\right]\lr{\pdeone{\cp}{T}\cdot\x}}
       {\displaystyle \lr{\cp\cdot\x}^{2}}.
\end{align}
Note that 
\begin{align}
  \mathbf{J}_{\omega} = \pdeone{\rr}{\x}
\end{align}
and $\mathbf{e}_{j}$ is as vector with $N$ entries containing all 
zeros except at entry $j$ which has value $1$.

\end{document}



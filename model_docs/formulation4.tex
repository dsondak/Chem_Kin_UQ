\subsection{Transition Matrix Formulaton}
We are once again re-thinking the inadequacy formulation.  The inadequacy 
formulation takes the usual form,
\begin{align}
  \odeone{\mathbf{x}}{\time} = \mathbf{f}\lr{\mathbf{x}, T} + \mathbf{h}\lr{\mathbf{x}, T}.
\end{align}
We consider the inadequacy model $\mathbf{h}$ in the form,
\begin{align}
  \mathbf{h}\lr{\mathbf{x}, T} = A^{T}\widetilde{\mathbf{x}} + B\widehat{\mathbf{x}}.
  \label{eq:trans_inad}
\end{align}
In~\eqref{eq:trans_inad}, $A$ is the species to atoms transition matrix and 
$B$ is a matrix whose columns span the nullspace of $A$.  The quantities 
$\widetilde{\mathbf{x}}\in\mathbb{R}^{\na}$ and 
$\widehat{\mathbf{x}}\in\mathbb{R}^{\ns-\na}$ are random variables. 
Note that we must find a way to build the temperature dependence into 
these perturbations.  Moreover, these perturbations may have a nonlinear 
form; they are not necessarily the species and atoms from the reduced model.  
The transition matrix $A$ has size $\na\times\ns$. 
For the \ce{H2}/\ce{O2} system, using the reduced species set, the 
transition matrix is
\begin{align}
  A = 
  \begin{bmatrix}
    2 & 0 & 1 & 0 & 1 & 1 & 2 \\
    0 & 2 & 0 & 1 & 1 & 2 & 1
  \end{bmatrix}.
\end{align}
Note that the total number of atoms of each type is given by 
\begin{align}
  \mathbf{a}_{\textrm{tot}} = A\mathbf{x}.
\end{align}
We know that the total number of atoms is conserved.
Thus
\begin{align}
  0 = \odeone{\mathbf{a}_{\textrm{tot}}}{\time} = 
  A \odeone{\mathbf{x}}{\time} = A\mathbf{f}
\end{align}
Solutions to $A\mathbf{x} = 0$ are given by 
\begin{align}
  \mathbf{x} = 
  \begin{bmatrix} -1/2 \\ 0 \\ 1 \\ 0 \\ 0 \\ 0 \\ 0 \end{bmatrix}x_{3} 
  + 
  \begin{bmatrix} 0 \\ 1/2 \\ 0 \\ 1 \\ 0 \\ 0 \\ 0 \end{bmatrix}x_{4} 
  + 
  \begin{bmatrix} -1/2 \\ \displaystyle -1/2 \\ 0 \\ 0 \\ 1 \\ 0 \\ 0 \end{bmatrix}x_{5} 
  + 
  \begin{bmatrix} -1/2 \\ -1 \\ 0 \\ 0 \\ 0 \\ 1 \\ 0 \end{bmatrix}x_{6} 
  + 
  \begin{bmatrix} -1 \\ -1/2 \\ 1 \\ 0 \\ 0 \\ 0 \\ 1 \end{bmatrix}x_{7}.
\end{align}
The dimension of the null space is therefore $5$.  It may be that in general 
the dimension of the nullspace of $A$ is $\ns - \na$ but I don't have a 
proof of this at this exact moment.

The form of~\eqref{eq:trans_inad} is such that we are perturbing the 
reduced model in the direction of range of $A$ as well as the direction 
of the null space of $A$.  The first perturbation (in the direction 
of the range of $A$) may result in violations of conservation of atoms.
To see this premultiply~\eqref{eq:trans_inad} by $A$.
\begin{align}
  &A\odeone{\mathbf{x}}{\time} = A\mathbf{f} + A\mathbf{h} \\
  \Rightarrow &\odeone{\mathbf{a}}{\time} = A\mathbf{f} + A\mathbf{h} \\
              &= \underbrace{A\mathbf{f}}_{0} + 
                 AA^{T}\widetilde{\mathbf{x}} + 
                 \underbrace{AB}_{0}\widehat{\mathbf{x}} \\
              &= AA^{T}\widetilde{\mathbf{x}} \neq 0.
\end{align}

We now consider conservation of atoms.  It may be necessary to perturb
the total atoms to guarantee that atoms are conserved.  We introduce 
the virutal species $\xv$ to help enforce conservation.  Thus, 
\begin{align}
  \mathbf{x}_{\textrm{tot}} = A\mathbf{x} + \xv
\end{align}
where $\xv$ is a vector containing the virtual species and acting 
as a correction to guarantee conservation.  Now we have 
\begin{align}
  0 = \odeone{\mathbf{x}_{\textrm{tot}}}{\time} &= A\odeone{\mathbf{x}}{\time} + 
       \odeone{\xv}{\time} \\
    &= A\lr{\mathbf{f} + \mathbf{h}} + \odeone{\xv}{\time}.
\end{align}
Thus, the virtual species must evolve according to 
\begin{align}
  \odeone{\xv}{\time} = -A\lr{\mathbf{f} + \mathbf{h}}.
\end{align}

We could rewrite our system in a more compact form as 
\begin{align}
  \odeone{\mathbf{x}}{\time} &= \mathbf{f} + M\pert \label{eq:gen_inad} \\
  \odeone{\xv}{\time} &= -A\lr{\mathbf{f} + M\pert} \label{eq:xv_evo}.
\end{align}
Note that~\eqref{eq:xv_evo} can be written as 
\begin{align}
  \odeone{\xv}{\time} = -AM\pert.
\end{align}

At this point we have conservation.  We are still missing a form 
for the perturbations $\pert$ and we still have not said anything 
about positivity.  Positivity requires,
\begin{align}
  x_{i}    &: \quad f_{i} + M_{ij}\pertk{j} \geq 0 
       \quad \textrm{for} \quad x_{i} = 0 \\
  x_{k}^{v}&: \quad -A_{kj}M_{jl}\pertk{l} \geq 0 
       \quad \textrm{for} \quad \xvk{k} = 0
\end{align}
If we choose $M_{ij}\pertk{j} = \alpha_{i}f_{i} + \lambda_{i}x_{i}$ 
then we guarantee that $M_{ij}\pertk{k} \to \epsilon_{i}$ as 
$x_{i} \to 0$.  Then 
\begin{align}
  \odeone{x_{i}}{\time} &= f_{i} + \lr{\alpha_{i}f_{i} + \lambda_{i}x_{i}} \\
  \odeone{x_{k}^{v}}{\time} &= -A_{kl}\lr{\alpha_{l}f_{l} + \lambda_{l}x_{l}}.
\end{align}
If $\alpha_{l} = \alpha$ then 
\begin{align}
  \odeone{x_{k}^{v}}{\time} &= -A_{kl}\lambda_{l}x_{l}.
\end{align}
If we require $\lambda_{l} \leq 0$ then the virtual species 
will become active.  However, we would still need to provide 
a mechanism by which the virtual species can impact the 
actual species.

\begin{tcolorbox}[arc is angular, title=Some thoughts...]
  \begin{itemize}
    \item Think of a way to put a contribution of the 
          virtual species into the actual species.
    \item Another formulation using some kind of progress 
          variable?  This variable would be off at 
          equilibria and on otherwise.
  \end{itemize}
\end{tcolorbox}
